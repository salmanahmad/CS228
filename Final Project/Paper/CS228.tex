
\documentclass{article}


% The file ijcai11.sty is the style file for IJCAI-11 (same as ijcai07.sty).
\usepackage{ijcai11,times,graphicx,amsmath,float,multirow,bbm,amssymb}


\title{Gesture Alignment Using Hidden Markov Models}
\author{
Andrew Hershberger
\quad Salman Ahmad \\
\{andrew.hershberger, saahmad\}@stanford.edu
\\\\
Stanford University\\
CS228: Probabilistic Graphical Methods - Winter 2011\\
}

\begin{document}

\maketitle

\begin{abstract}

Gesture recognition gaining interest in many domains. A common issue when
learning from multiple training gestures is accounting for noise and different
durations and speeds. This paper presents an algorithm that uses Hidden Markov
Models to align training gestures and learn the hidden canonical gesture that
they represent. The algorithm was used to align motion capture data from an
Xbox Kinect. This paper present initial results along with a discussion of
current limitations and paths for future work.

\end{abstract}

\section{Introduction}

Gesture recognition is becoming increasingly important in many fields from
gaming to user interface design. Since it is difficult to manually encode
gestures declaratively, there is a lot of interest in applying learning
techniques to train a classifier that can recognize gestures from motion
capture data.

A common problem with this approach is that the training examples are often
different durations and different speeds. This paper provides an algorithm
that aligns gestures based on the important actions that take place - e.g. the
start of a wave or the midpoints in a jump. Additionally, the algorithm
learns a canonical representation of the gesture that can be used for
classification of new data.

\begin{figure}
\begin{centering}
\includegraphics[width=0.4\columnwidth]{figures/control_points.pdf}

\caption{The location twenty control points that are tracked by the gesture alignment
algorithm.\label{figure:control_points}}

\end{centering}
\end{figure}

The algorithm was evaluated using motion capture information from an Xbox
Kinect. The raw RGBZ output was converted to \emph{(x,y,z)} positions of 20
different control points on the human body as shown in Figure
\ref{figure:control_points}. An alignment hidden markov model was then used to
align the different gestures.

The rest of this paper presents related work in this area, the graphical model
used to encode the independencies of the data, a discussion of the algorithm,
results, an analysis of current limitations, and logical areas of future work.


\section{Related Work}

ANDREW:

Dynamic Time Warping \cite{Listgarten2005}

Learning Control \cite{Coates2008}

Probabilistic Graphical Models: Principles and Techniques \cite{Koller2009}

Dynamic Bayesian Networks: Representation, Inference and Learning \cite{Muphy2002}

Hidden Conditional Random Fields for Gesture Recognition \cite{Wang2006}


\section{Graphical Model}

The algorithm uses an alignment HMM to align observed 

\begin{figure}
\begin{centering}
\includegraphics[width=0.65\columnwidth]{figures/model_tau_unobserved.pdf}

\caption{The graphical model using for gesture alignment. $X$ represents the
hidden, ``canonical'' gesture. $Y$ represents the observed gesture from the
training set. $T$ represents an indexed mapping between the observed gesture
to the canonical gesture. The dark nodes are observed.
\label{figure:model_tau_unobserved}}

\end{centering}
\end{figure}



\begin{figure}
\begin{centering}
\includegraphics[width=0.65\columnwidth]{figures/model_tau_observed.pdf}

\caption{An example graphical model once $T$ has been learned. In this case,
the second and third frames from the observed gesture maps to the second and
fifth frame of the canonical gesture. The dark nodes are observed.
\label{figure:model_tau_observed}}

\end{centering}
\end{figure}



images of the model before and after dynamic programming approach to DTW

It's an alignment HMM

\section{Algorithm}

ANDREW:

What we did: EM, DTW

Optimizing the algorithm when calculating q(:,:), tau;

Different smoothing

no prior knowledge of optimal trajectory

not using a bias function

Tried different allowed step sizes for d

\section{Results}

SALMAN:




\begin{figure}
\begin{centering}
\includegraphics[width=\columnwidth]{figures/kick_aligned.png}

\vspace{0.1in}

\includegraphics[width=\columnwidth]{figures/kick_unaligned.png}

\caption{Motion capture data of a person performing a kick. Top: Data that has
been aligned with our algorithm. Bottom: The Original, unaligned data. Both
images were taken at the same time offset. \label{figure:kick}}

\end{centering}
\end{figure}





\begin{figure}
\begin{centering}
\includegraphics[width=\columnwidth]{figures/jump.png}

\caption{A failure case of our implementation. The algorithm does not have
domain specific information about the dynamics of the real world, for example,
gravity. The above data was taken from a person jumping. To align the data,
the algorithm ``freezes'' the person in mid-air when this is obviously
physically impossible. \label{figure:jump}}

\end{centering}
\end{figure}









Figures, writeup (we are geniuses).

\section{Discussion and Future Work}

ANDREW:

Why didn't it work very well? Smoothing made things worse.

Use different smoothing

Add prior knowledge of optimal trajectory: can incorporate effects of gravity - don't want things to hang in mid air.

Application to classification problem

Add more data (training data)

Add features to detect particular aspects of gestures.

Detect orientation differences

\section{Conclusion}

This paper presents a method to perform gesture alignment using a Hidden
Markov Model. The algorithm was shown to be able to align certain gestures and
learn a canonical gesture. The method was applied to motion capture data that
was extracted from RGBZ images taken from an Xbox Kinect.

While the findings were some what promising it failed to work on a diverse set
of gestures. There are obvious areas for future work. First, the model should
incorporate our prior knowledge of the ideal gesture. For example, it would
certainly help to encode that during a kick, one of the legs will be
accelerating while the rest of the body stays still. Second, the algorithm
should incorporate a dynamics model of the real world. This will allow the
method to better deal with physical phenomenons like gravity.





\bibliographystyle{named}
\bibliography{CS228}

\end{document}
